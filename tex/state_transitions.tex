\label{state_transitions}

In questa sezione mostro il motivo per cui non sempre la composizione ha
l'effetto sperato.
Il $Monad$ per lo $State\ transitions$ difatti rappresenta un esempio pratico
in cui, dato un $Monad\ M$, la composizione generalizzata non è ciò che ci
aspettiamo.\newline

L'$ADT$ per lo $state\ transitions$
\begin{align*}
  \bm{data}\ State\ s\ a = ST(s \to (a,\ s))
\end{align*}
Difatti il monade per le transizioni di stato è definito
\begin{align*}
  \classdef{\bm{instance}\ Functor\ (State\ s)\ \bm{where}}
  map\ f\ (ST\ x)\ &= ST(\lambda s \to \bm{let}\ (s',\ x) = st\ s\ \bm{in}\ (s',\ f\ x))\\
  \classdef{\bm{instance}\ Premonad\ (State\ s)\ \bm{where}}
  unit\ x &= ST(\lambda s \to (s,\ x))\\
  \classdef{\bm{instance}\ Monad\ (State\ s)\ \bm{where}}
  join\ (ST\ m) &= ST(\lambda s \to \bm{let}\ (s',\ ST\ m') = m\ s\ \bm{in}\ m'\ s')\\
\end{align*}
Possiamo notare come lo $State$ possa essere definito in termini di
\textit{Monad Composition} tra $Reader$ e $Writer$.
Difatti possiamo comporre lo $State$ in questo modo, $State\ s\ a = (s \to)\ .\ (Writer\ s)\ a$.
Oppure mediante il framework della sezione \ref{general_framework_for_composition},
$\bm{data}\ State\ s\ a = DComp\ (s \to)\ (Writer\ s)\ a$.\newline

Questo nuovo $Monad$ non è equivalente allo $State$ definito sopra, difatti nel caso
$unit$ il monade composto ritorna, dato un qualsiasi $s$ lo $zero$ del $Monoid$
$s$, inoltre richiede appunto che $s$ sia $Monoid$.\newline

Potremmo quindi creare un caso specifico per combinare un qualsiasi $Monad$ con
lo \textit{State Transitions}, prima di tutto definiamo il suo tipo, sarà generico
in $m$
\begin{align*}
  \bm{data}\ StateM\ m\ s\ a = STM(s \to m\ (a,\ s))
\end{align*}
Poi istanziamolo ad essere un monade
\begin{align*}
  \classdef{\bm{instance}\ Monad\ m \implies Functor\ (StateM\ m\ s)\ \bm{where}}
  map\ f\ (STM\ xs)\ &= STM(\lambda s \to [(s',\ f\ x)\ |\ (s',\ x) \leftarrow xs\ s])\\
  \classdef{\bm{instance}\ Monad\ m \implies Premonad\ (StateM\ m\ s)\ \bm{where}}
  unit\ xs &= STM(\lambda s \to [(s,\ x)])\\
  \classdef{\bm{instance}\ Monad\ m \implies Monad\ (StateM\ m\ s)\ \bm{where}}
  join\ (STM\ xss) &= STM(\lambda s \to [(s'',\ x)\ |\ (STM\ xs,\ s') \leftarrow xss\ s, \\
                   &\qquad\qquad\qquad\qquad\qquad\qquad(s'',\ x) \leftarrow xs\ s'])\\
\end{align*}
Ora riusciamo a comporre un qualsiasi $Monads$ con il sistema di \textit{State Transitions}
, notiamo però che abbiamo creato un sistema di composizione generalizzata ad hoc.
Quindi non sfruttiamo il framework precedentemente creato.\newline

Una spiegazione a questo fatto è che la composizione di monadi generata dal
nostro framework non riesce ad esprimere la relazione che intercorre tra i
componenti del monade composto.
Questa spiegazione, istanziata al nostro caso, in particolare non esprime
l'equivalenza tra il parametro in input al $Reader\ Monad$ con l'output prodotto
nel $Writer\ Monad$.\newline

Steele \cite{steele0}, prova a definire un meccanismo generico anche per gli $ADT$
che hanno queste caratteristiche di relazioni tra monadi composti.
Utilizza un sistema di pseudomonadi per riuscire a comporre monadi lasciando il monade
arbitrario non più ai "lati" della composizione ma al "centro", non approfondirò
questo punto all'interno di questo report.