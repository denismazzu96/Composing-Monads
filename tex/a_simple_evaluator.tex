\section{A Simple Evaluator}
\label{a_simple_evaluator}

\subsection{Abstract Syntax} %--------------------------------------------------
\label{abstract_syntax}

La funzione di valutazione riceverà in input espressioni generate dalla seguente
grammatica:

\begin{figure}[h]
  \centering
  \footnotesize % adapt to your page size
  \begin{haskellsyntax}
  \mv{Expr} \STO \mv{Const} \ \mv{Value}
                  \REM{Valore costante}
            \SOR \mv{Var} \ \mv{Name}
                  \REM{Variabile in memoria}
            \SOR \mv{Expr} + \mv{Expr}
                  \REM{Addizione tra espressioni}
            \SOR \mv{Trace} \ \mv{String} \ \mv{Expr}
                  \REM{Struttura di log}
  \\
  \mv{Env} \STO [(\mv{Name}, \mv{Value})]
                \REM{Ambiente di mapping delle variabili}
  \\
  \mv{Name} \STO \mv{String}
  \\
  \mv{Value} \STO \mv{Int}
  \end{haskellsyntax}%

\end{figure}

Per mantere l'esempio semplice ho ridotto al minimo le espressioni generabili
nel linguaggio, permettendo solamente costanti, variabili, addizioni ed un
semplice meccanismo di log.\newline

Si noti come questa grammatica permetta comunque l'implementazione di alcune
caratteristiche basilari come: la gestione degli errori per variabili
sconosciute, un sistema di output e la gestione dell'accesso delle variabili nel
sistema.

\subsection{Implementation} %---------------------------------------------------
\label{implementation}

Per la prima implementazione della funzione di valutazione utilizzo i seguenti
tipi astratti:

\begin{figure}[h]
  \centering
  \footnotesize % adapt to your page size
  \begin{haskellsyntax}
  \textbf{data} \ Error \ a &= Error \ String \sor Ok \ a
  \\
  \textbf{data} \ Writer \ s \ a &= (s, a)
    \REM{con s $::$ \textit{Monoid}, vedi \ref{monoid}}

  \end{haskellsyntax}%

\end{figure}

Con questi due tipi riusciamo a codificare la gestione degli errori e la
gestione del log, per la codifica dell'\textit{Enviroments} definiamo
semplicemente la funzione di valutazione, che riceverà in input lo stato
iniziale e ritornerà un valore (se nessun errore è occorso) e un output,
possibilmente vuoto.

\begin{align*}
  eval :: Expr \to Env \to Writer \ [String] \ (Error \ Value)
\end{align*}

Già dal tipo della funzione \textit{eval} si può notare come sia poco pulito,
difatti il tipo di ritorno è molto complesso dato che deve gestire ben due
caratteristiche non banali, la gestione dell'output e degli errori.
Ora passiamo all'implementazione dell'interprete.

\begin{figure}[H]
  \centering
  \footnotesize % adapt to your page size
  \begin{haskellsyntax}
  eval \ (Const \ v) \ env &= ([\ ], \ Ok \ a)\\
  eval \ (Var \ x) \ env   &= ([\ ], \ lookup \ env \ x)\\
  eval \ (e_0 + e_1) \ env  &= \textbf{case} \ eval \ e_0 \ env \ \textbf{of}\\
                            {}& \qquad (out_0, \ Error \ s) \to ([\ ], \ Error \ s)\\
                            {}& \qquad (out_0, \ Ok \ v_0) \quad \to \textbf{case} \ eval \ e_1 \ env \ \textbf{of}\\
                            {}& \qquad \qquad (out_1, \ Error \ s) \to ([\ ], \ Error \ s)\\
                            {}& \qquad \qquad (out_1, \ Ok \ v_1) \quad \to (out_0 \doubleplus out_1, v_0 \textbf{+} v_1)\\
  eval \ (Trace \ s \ e) \ env  &= ([s], \ eval \ e \ env)
  \end{haskellsyntax}%
\end{figure}

Prima di procedere notiamo l'utilizzo della funzione \textit{lookup} definita come segue.

\begin{figure}[H]
  \centering
  \footnotesize % adapt to your page size
  \begin{haskellsyntax}
  lookup &:: Env \to Name \to Error\ Value \\
  lookup \ [\ ] \ x &= Error \ "Value \ unbounded"\\
  lookup \ [(name, \ value):xs] \ x &= \textbf{if} \ name = x \
                                          \textbf{then} \ Ok \ value \
                                          \textbf{else} \ lookup \ xs \ x
  \end{haskellsyntax}%
\end{figure}

Come possiamo notare questo interprete è complicato e necessita di esplicitare
tutti i passi computazionali esponendolo ad errori.
Il codice risulta meno mantenibile e ancora meno estensibile, un qualsiasi
cambiamento, o richiesta di nuova \textit{feature}, richiederebbe la modifica
di quasi la totalità delle righe di codice.
In un linguaggio imperativo banale l'implementazione di queste caratteristiche
è un task molto semplice e non richiede particolari competenze, i
\textit{Monads} serviranno proprio per portare questa facilità d'esecuzione
dentro ad un linguaggio funzionale puro.


\newpage
