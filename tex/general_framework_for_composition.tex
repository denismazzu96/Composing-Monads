\pagebreak
\section{General Framework for composition}
\label{general_framework_for_composition}

In questa sezione verranno definite una serie di classi per la generazione di
un framework per la composizione di monadi.
Dal precedente capitolo \ref{composing_monads}, definiremo una serie di classi e
funzioni per poter tipare correttamente le operazioni necessarie e definire
finalmente il tipo di dato astratto per la composizione generalizzata.
Le classi saranno implementate in sintassi Gofer per permettere un migliore
ragionamento e in Scala per avere esmpio concreti.\newline

Abbiamo già visto come non sia possibile definire la composizione che riceva
due $Monads$ arbitrari.
La prossima migliore opzione è di fissare uno dei due componenti (della
composizione) e permettere all'altro componente di variare dentro ad una
famiglia di differenti Monads.
Come questo accada sarà mostrato con qualche esempio nella sezione
\ref{some_examples}, prima definiamo il framework.\newline

\subsection{Representing Functors, Premonads and Monads}
Vediamo velocemente come si rappresentano i tipi costruttori base.
\begin{align*}
  \classdef{\bm{class}\ Functor\ f\ \bm{where}}
  map &:: (a \to b) \to (f\ a \to f\ b)\\\\
  \classdef{\bm{class}\ Functor\ m \Rightarrow Premonad\ m\ \bm{where}}
  unit &:: a \to m\ a\\\\
  \classdef{\bm{class}\ Premonad\ m \Rightarrow Monad\ m\ \bm{where}}
  join &:: m\ (m\ a) \to m\ a
\end{align*}
Con $\implies$ si intende la relazione di $subtyping$, ovvero la classe a sinistra
di $\implies$ è sotto classe di quella definita a sinistra.

\begin{lstlisting}[style=myScalastyle, caption=Functor Premonad and Monad Traits]
  trait Functor[F[_]]{
    def map[A,B](ma:F[A])(f:(A) => B):F[B]
  }
  trait Premonad[M[_]] extends Functor[M]{
    def unit[A](a:A):M[A]
  }
  trait Monad[M[_]] extends Functor[M]{
    // come il join operator
    def flatten[A](mm:M[M[A]]):M[A]
    // come il bind operator, Scala defisce cosi' il Monad
    def flatMap[A,B](ma:M[A])(f:(A)=>M[B]) = flatten(map(ma)(f))
  }
\end{lstlisting}

\subsection{Composition Construction}
\label{composition_construction}

Per descrivere la composizione di $Functors$ e $Premonads$ utilizziamo le
definizioni
\begin{align*}
  mapC &:: (Functor\ f,\ Functor\ g) \implies (a \to b) \to
  (f \ (g \ a) \to f\ (g\ b))\\
  mapC &= map\ .\ map\\
  unitC &= (Premonad\ f,\ Premonad\ g) \implies a \to f\ (g\ a)\\
  unitC &= unit\ .\ unit
\end{align*}
Queste due firme sono del tipo corretto per essere utilizzate come definizione
di funtore e premonade.
Tuttiavia, non possono essere utilizzate direttamente come istanze di funtore (e
premonade) dato che non c'è modo di definire una classe $(f\ .\ g)$ istanza
di $Functor$ o $Premonad$ o $Monad$.
Precisamente, non c'è nulla nel tipo dell'espressione $f\ (g\ x)$ che indichi
a quale costruzione è destinato.
Per evitare questa ambiguità, definiamo un costruttore $c$ per ogni
\textit{composition constructions} con l'intenzione che $c\ f\ g\ x$ sia la
composizione $f\ (g\ a)$, identificando quindi la costruzione utilizzata.
\label{composer}
\begin{align*}
  \classdef{\bm{class}\ Composer\ c\ \bm{where}}
  open &:: c\ f\ g\ x \to f\ (g\ x)\\
  close &:: f\ (g\ x) \to c\ f\ g\ x
\end{align*}
Ora proceidamo a definire le istanze di $Functor$ e $Premonad$ per i tipi compositi
, nota che queste definizioni sono \textit{general purpose}, quindi non necessitano di essere
ridefinite ogni volta.
\begin{align*}
  \classdef{\bm{instance}\ (Composer\ c\,\ Functor\ f,\ Functor\ g) \implies Functor\ (c\ f\ g)\  \bm{where}}
  map\ f &= close\ .\ mapC\ f\ .\ open\\
  \classdef{\bm{instance}\ (Composer\ c\,\ Premonad m,\ Premonad n)}
  \classdef{\implies Premonad\ (c\ m\ n)\  \bm{where}}
  unit &= close\ .\ unitC
\end{align*}
\begin{lstlisting}[style=myScalastyle, caption=Natural map and unit operator]
  implicit def premonadsCompose[M[_],N[_]]
    (implicit m:Monad[M], n:Monad[N])
    = new Premonad[({type MN[x] = M[N[x]]})#MN]{
      def map[A,B](ma:M[N[A]])(f: (A) => B)
        = m.map(ma)(n.map(_)(f))
      def unit[A](a: A) =  m.unit(n.unit(a))
  }
  \end{lstlisting}
Le prossime definizioni saranno utilizzate molteplici volte nelle sezioni
seguenti.
Il loro scopo è di convertire funzioni per il $join$ di composizioni nella forma
equivalente usando un'istanza $c$ della classe $Composer$.
\begin{align*}
  wrap &:: (Composer\ c, Functor\ m, Functor\ n) \implies\\
       &\quad (m\ (n\ (m\ (n\ a))) \to m\ (n\ a)) \to\\
       &\qquad (c\ m\ n\ (c\ m\ n\ a) \to c\ m\ n\ a)\\
  wrap\ j &= close\ .\ j\ .\ mapC\ open\ .\ open
\end{align*}
In special modo le prossime due funzioni servono a fornire un modo per racchiudere
la computazione di un componente dentro il \textit{composite monad}.
\label{left_and_right}
\begin{align*}
  right &:: (Composer\ c,\ Premonad\ f) \implies g\ a \to c\ f\ g\ a\\
  right &= close\ .\ unit\\\\
  left &:: (Composer\ c,\ Functor\ f,\ Premonad\ g) \implies f\ a \to c\ f\ g\ a\\
  left &= close\ .\ map\ unit
\end{align*}

\subsection{Programming Prod Construction}
\label{programming_prod_construction}
Creaiamo un'istanza del $Composer$ sopra definito per il costruttore $prod$
\begin{align*}
  \classdef{\bm{data}\ PComp\ f\ g\ x\ =\ PC\ (f\ (g\ x))}
  \classdef{\bm{instance}\ Composer\ PComp\ \bm{where} }
  open\ (PC\ x)\ &= x\\
  close &= PC
\end{align*}
La costruzione richiede lei stessa la funzione $prod$ come definita
\begin{align*}
  \classdef{\bm{class}\ (Monad\ m,\ Premonad\ n) \implies PComposable\ m\ n\ \bm{where}}
  prod &:: n\ (m\ (n\ a)) \to m\ (n\ a)
\end{align*}
\begin{lstlisting}[style=myScalastyle, caption=Prod Constructor]
  trait PComposable[M[_],N[_]]{
    def M:Monad[M]
    def N:Premonad[N]
    def prod[A](nm:N[M[N[A]]]):M[N[A]]
  }
\end{lstlisting}
Notiamo come le richieste imposte specifichino per $n$ solamente l'appartenenza
alla classe dei $Premonads$, mentre per $m$ si richiede di essere un $Monads$
\begin{align*}
  joinP &:: (Pcomposable\ m\ n) \implies m\ (n\ (m\ (n\ a))) \to m\ (n\ a)\\
  joinP &= join\ .\ map\ prod
\end{align*}
Finalmente possiamo definire che $PComp$ sia un $Monad$
\begin{align*}
  \classdef{\bm{instance}\ PComposable\ m\ n \implies Monad\ (PComp\ m\ n)\ \bm{where}}
  join &= wrap\ joinP
\end{align*}
\begin{lstlisting}[style=myScalastyle, caption=PComposable is Monad]
  implicit def monadPCompose[M[_],N[_]]
    (implicit m:Monad[M], n:Monad[N], p:PComposable[M,N])
    = new Monad[({type MN[x] = M[N[x]]})#MN]{
      def flatten[A](mnmn:M[N[M[N[A]]]])
        = m.flatten(m.map(mnmn)(p.prod(_)))
  }
  \end{lstlisting}

\subsection{Programming Dorp Construction}
\label{programming_dorp_construction}
Come per il caso precedente, procediamo in maniera simile.
Creaiamo un'istanza del $Composer$ sopra definito per il costruttore $dorp$
\begin{align*}
  \classdef{\bm{data}\ DComp\ f\ g\ x\ =\ DC\ (f\ (g\ x))}
  \classdef{\bm{instance}\ Composer\ DComp\ \bm{where} }
  open\ (DC\ x)\ &= x\\
  close &= DC
\end{align*}
La costruzione richiede lei stessa la funzione $dorp$ come definita
\begin{align*}
  \classdef{\bm{class}\ (Premonad\ m,\ Monad\ n) \implies DComposable\ m\ n\ \bm{where}}
  dorp &:: m\ (n\ (m\ a)) \to m\ (n\ a)
\end{align*}
\begin{lstlisting}[style=myScalastyle, caption=Dorp Constructor]
  trait DComposable[M[_],N[_]]{
    def M:Premonad[M]
    def N:Monad[N]
    def dorp[A](nm:M[N[M[A]]]):M[N[A]]
  }
\end{lstlisting}
Notiamo come le richieste imposte specifichino per $m$ solamente l'appartenenza
alla classe dei $Premonads$, mentre per $n$ si richiede di essere un $Monads$
\begin{align*}
  joinD &:: (DComposable\ m\ n) \implies m\ (n\ (m\ (n\ a))) \to \ (n\ a)\\
  joinD &= join\ .\ map\ dorp
\end{align*}
Finalmente possiamo definire che $DComp$ sia un $Monad$
\begin{align*}
  \classdef{\bm{instance}\ DComposable\ m\ n \implies Monad\ (DComp\ m\ n)\ \bm{where}}
  join &= wrap\ joinD
\end{align*}
\begin{lstlisting}[style=myScalastyle, caption=DComposable is Monad]
  implicit def monadDCompose[M[_],N[_]]
    (implicit m:Monad[M], n:Monad[N], d:DComposable[M,N])
    = new Monad[({type MN[x] = M[N[x]]})#MN]{
      def flatten[A](mnmn:M[N[M[N[A]]]])
        = m.flatten(m.map(mnmn)(d.dorp(_)))
  }
  \end{lstlisting}
\pagebreak
\subsection{Programming Swap Construction}
\label{programming_swap_construction}
Anche questo costruttore molto simile ai casi precedenti.
Creaiamo un'istanza del $Composer$ sopra definito per il costruttore $swap$
\begin{align*}
  \classdef{\bm{data}\ SComp\ f\ g\ x\ =\ SC\ (f\ (g\ x))}
  \classdef{\bm{instance}\ Composer\ SComp\ \bm{where} }
  open\ (SC\ x)\ &= x\\
  close &= SC
\end{align*}
La costruzione richiede lei stessa la funzione $swap$ come definita
\begin{align*}
  \classdef{\bm{class}\ (Premonad\ m,\ Monad\ n) \implies SComposable\ m\ n\ \bm{where}}
  swap &:: n\ (m\ a) \to m\ (n\ a)
\end{align*}
\begin{lstlisting}[style=myScalastyle, caption=Swap Constructor]
  trait SComposable[M[_],N[_]]{
    def M:Monad[M]
    def N:Monad[N]
    def swap[A](nm:N[M[A]]):M[N[A]]
  }
\end{lstlisting}
Notiamo come le richieste imposte specifichino sia per $m$ che per $n$ l'appartenenza
alla classe dei $Monads$, diversamente dai casi precedenti
\begin{align*}
  joinS &:: (SComposable\ m\ n) \implies m\ (n\ (m\ (n\ a))) \to \ (n\ a)\\
  joinS &= map\ join\ .\ join\ .\ map\ swap
\end{align*}
Finalmente possiamo definire che $SComp$ sia un $Monad$
\begin{align*}
  \classdef{\bm{instance}\ SComposable\ m\ n \implies Monad\ (SComp\ m\ n)\ \bm{where}}
  join &= wrap\ joinS
\end{align*}
\begin{lstlisting}[style=myScalastyle, caption=SComposable is Monad]
  implicit def monadSCompose[M[_],N[_]]
    (implicit m:Monad[M], n:Monad[N], s:SComposable[M,N])
    = new Monad[({type MN[x] = M[N[x]]})#MN]{
      def flatten[A](mnmn:M[N[M[N[A]]]])
        = m.map(m.flatten(m.map(mnmn)(s.swap(_))))(n.flatten(_))
  }
  \end{lstlisting}
Inoltre possiamo notare come un $SComposable$ sia egli stesso un $PComposable$ e
$DComposable$, questa definizione mostra come un costruttore $prod$ e $dorp$
possa sempre essere derivato da una definizione di $swap$
\begin{align*}
  \classdef{\bm{instance}\ SComposable\ m\ n \implies PComposable\ m\ n\ \bm{where}}
  prod &= map\ join\ .\ swap\\
  \classdef{\bm{instance}\ SComposable\ m\ n \implies DComposable\ m\ n\ \bm{where}}
  dorp &= map\ join\ .\ swap\\
\end{align*}

\pagebreak
\subsection{Some Examples}
\label{some_examples}

\subsubsection*{Reader}
Il Monade $Reader$ è un costruttore del tipo $(r\ \to)$ che mappa per ogni tipo
$a$ alla funzione del tipo $r \to a$, la sua strutture  è definita come segue

\begin{align*}
  \classdef{\bm{instance}\ Functor\ (r \to)\ \bm{where}}
  map\ f\ g\ &= f\ .\ g\\
  \classdef{\bm{instance}\ Premonad\ (r \to)\ \bm{where}}
  unit\ x\ y\ &= x\\
  \classdef{\bm{instance}\ Monad\ (r \to)\ \bm{where}}
  join\ f\ x\ &= f\ x\ x\\
\end{align*}
Questo caso combineremo un qualsiasi Monade $n$ con il monade $Reader$ usando il
costruttore $dorp$
\begin{align*}
  \classdef{\bm{instance}\ DComposable\ (r \to)\ n\ \bm{where}}
  dorp\ m\ r\ &=\ [\ g\ r\ |\ g \leftarrow m\ r\ ]
\end{align*}

\subsubsection*{Writer}
Il \textit{Writer Monad} invece è un Monade che descrive programmi che producono
sia valori che output in maniera parallela.
Inoltre questo Monade assume che il parametro fisso $s$ sia un $Monoid$, vedi
sezione \ref{monoid}.
\begin{align*}
  \classdef{\bm{instance}\ Functor\ (Writer\ s)\ \bm{where}}
  map\ f\ (Result\ s\ a)\ &= Result\ s\ (f\ a)\\
  \classdef{\bm{instance}\ Monoid\ s \implies Premonad\ (Writer\ s)\ \bm{where}}
  unit\ &= Result\ zero\\
  \classdef{\bm{instance}\ Monoid\ s \implies Monad\ (Writer\ s)\ \bm{where}}
  join\ (Result\ s\ (Result\ t\ x)) &= Result\ (add\ s\ t)\ x\\
\end{align*}
Questo Monade è un esempio di come si utilizza il costrutto $swap$ per combinare
un qualsiasi Monade
\begin{align*}
  \classdef{\bm{instance}\ SComposable\ (Writer\ s)\ n\ \bm{where}}
  swap\ (Result\ s\ m) &=\ [\ Result\ s\ a\ |\ a \leftarrow m\ ]
\end{align*}

\pagebreak
\subsubsection*{Maybe}
Il \textit{Maybe Monad} invece è un Monade che descrive programmi che producono
non sempre valori corretti.
Molto simile al monade $Error$.
\begin{align*}
  \classdef{\bm{instance}\ Functor\ Maybe\ \bm{where}}
  map\ f\ (Just\ x)\ &= Just\ (f\ x))\\
  map\ f\ (Nothing)\ &= Nothing\\
  \classdef{\bm{instance}\ Premonad\ Maybe\ \bm{where}}
  unit\ &= Just\\
  \classdef{\bm{instance}\ Monad\ Maybe\ \bm{where}}
  join\ (Just\ m) &= m\\
  join\ Nothing &= Nothing\
\end{align*}
Questo Monade è un esempio di come si utilizza il costrutto $prod$ per combinare
un qualsiasi Monade
\begin{align*}
  \classdef{\bm{instance}\ PComposable\ m\ Maybe\ \bm{where}}
  prod\ (Just\ m) &=\ m\\
  prod\ Nothing &=\ [\ Nothing\ ]\\
\end{align*}

\begin{lstlisting}[style=myScalastyle, caption=Composing Maybe]
  implicit def MonadMaybeSComposable[M[_]]
    (implicit monadM: Monad[M])
    = new SComposable[M,Maybe]{
      def M = implicitly[Monad[Maybe]]
      def N = monadM
      def prod[A](nm: Maybe[M[A]]) = nm match{
        case Just(ma) => monadM.map(ma)(Maybe(_))
        case Nothing => monadM.unit(Nothing)
      }
  }
\end{lstlisting}

\pagebreak