\pagebreak
\section{General Framework for composition}
\label{general_framework_for_composition}

In questa sezione verranno definite una serie di classi per la generazione di
un framework per la composizione di monadi.
Dal precedente capitolo \ref{composing_monads}, definiremo una serie di classi e
funzioni per poter tipare correttamente le operazioni necessarie e definire
finalmente il tipo di dato astratto per la composizione generalizzata.
Le classi saranno implementate in sintassi Gofer per permettere di essere
codificate in un qualsiasi linguaggio funzionale.\newline

Abbiamo già visto come non sia possibile definire la composizione che riceva
due $Monads$ arbitrari.
La prossima migliore opzione è di fissare uno dei due componenti (della
composizione) e permettere all'altro componente di variare dentro ad una
famiglia di differenti Monads.
Come questo accada sarà mostrato con qualche esempio nella sezione
\ref{some_examples}, prima definiamo il framework.\newline

\subsection{Representing Functors, Premonads and Monads}
Vediamo velocemente come si rappresentano i tipi costruttori base.
\begin{align*}
  \classdef{\bm{class}\ Functor\ f\ \bm{where}}
  map &:: (a \to b) \to (f\ a \to f\ b)\\\\
  \classdef{\bm{class}\ Functor\ m \Rightarrow Premonad\ m\ \bm{where}}
  unit &:: a \to m\ a\\\\
  \classdef{\bm{class}\ Premonad\ m \Rightarrow Monad\ m\ \bm{where}}
  join &:: m\ (m\ a) \to m\ a
\end{align*}
Con $\implies$ si intende la relazione di $subtyping$, ovvero la classe a sinistra
di $\implies$ è sotto classe di quella definita a sinistra.

\subsection{Composition Construction}
\label{composition_construction}

Per descrivere la composizione di $Functors$ e $Premonads$ utilizziamo le
definizioni
\begin{align*}
  mapC &:: (Functor\ f,\ Functor\ g) \implies (a \to b) \to
  (f \ (g \ a) \to f\ (g\ b))\\
  mapC &= map\ .\ map\\
  unitC &= (Premonad\ f,\ Premonad\ g) \implies a \to f\ (g\ a)\\
  unitC &= unit\ .\ unit
\end{align*}
Queste due firme sono del tipo corretto per essere utilizzate come definizione
di funtore e premonade.
Tuttiavia, non possono essere utilizzate direttamente come istanze di funtore (e
premonade) dato che non c'è modo di definire una classe $(f\ .\ g)$ istanza
di $Functor$ o $Premonad$ o $Monad$.
Precisamente, non c'è nulla nel tipo dell'espressione $f\ (g\ x)$ che indichi
a quale costruzione è destinato.
Per evitare questa ambiguità, definiamo un costruttore $c$ per ogni
\textit{composition constructions} con l'intenzione che $c\ f\ g\ x$ sia la
composizione $f\ (g\ a)$, identificando quindi la costruzione utilizzata.
\begin{align*}
  \classdef{\bm{class}\ Composer\ c\ \bm{where}}
  open &:: c\ f\ g\ x \to f\ (g\ x)\\
  close &:: f\ (g\ x) \to c\ f\ g\ x
\end{align*}
Ora proceidamo a definire le istanze di $Functor$ e $Premonad$ per i tipi compositi
, nota che queste definizioni sono \textit{general purpose}, quindi non necessitano di essere
ridefinite ogni volta.
\begin{align*}
  \classdef{\bm{instance}\ (Composer\ c\,\ Functor\ f,\ Functor\ g) \implies Functor\ (c\ f\ g)\  \bm{where}}
  map\ f &= close\ .\ mapC\ f\ .\ open\\
  \classdef{\bm{instance}\ (Composer\ c\,\ Premonad m,\ Premonad n) \implies Premonad\ (c\ m\ n)\  \bm{where}}
  unit &= close\ .\ unitC
\end{align*}
Le prossime definizioni saranno utilizzate molteplici volte nelle sezioni
seguenti.
Il loro scopo è di convertire funzioni per il $join$ di composizioni nella forma
equivalente usando un'istanza $c$ della classe $Composer$.
\begin{align*}
  wrap &:: (Composer\ c, Functor\ m, Functor\ n) \implies\\
       &\quad (m\ (n\ (m\ (n\ a))) \to m\ (n\ a)) \to\\
       &\qquad (c\ m\ n\ (c\ m\ n\ a) \to c\ m\ n\ a)\\
  wrap\ j &= close\ .\ j\ .\ mapC\ open\ .\ open
\end{align*}
In special modo le prossime due funzioni servono a fornire un modo per racchiudere
la computazione di un componente dentro il \textit{composite monad}.
\label{left_and_right}
\begin{align*}
  right &:: (Composer\ c,\ Premonad\ f) \implies g\ a \to c\ f\ g\ a\\
  right &= close\ .\ unit\\\\
  left &:: (Composer\ c,\ Functor\ f,\ Premonad\ g) \implies f\ a \to c\ f\ g\ a\\
  left &= close\ .\ map\ unit
\end{align*}

\subsection{Programming Prod Construction}
\label{programming_prod_construction}

\subsection{Programming Dorp Construction}
\label{programming_dorp_construction}

\subsection{Programming Swap Construction}
\label{programming_swap_construction}

\subsection{Some Examples}
\label{some_examples}

\subsubsection*{Reader}

\subsubsection*{Writer}

\subsubsection*{Error}

\pagebreak