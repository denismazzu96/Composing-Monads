\label{scala_code}
Come richiesto in questa sezione inserirò degli $snippets$ di codice $Scala$
per la composizione generalizzata di monadi.
Non essendo pratico di $scala$ sono ricorso agli esempi definiti da Omura \cite{omura0}
e Milewski \cite{milewski0}.
Questo esempio ricrea il framework generalizzato per la composizione simile a quello
visto nella sezione \ref{general_framework_for_composition}.

\begin{lstlisting}[style=myScalastyle, caption=Functor and Monad Traits]
  trait Functor[F[_]]{
    def map[A,B](ma:F[A])(f:(A) => B):F[B]
  }
  // per l'esempio unisco la definizione di Monad e Premonad
  trait Monad[M[_]] extends Functor[M]{
    def unit[A](a:A):M[A]
    // come il join operator
    def flatten[A](mm:M[M[A]]):M[A]
    // flatMap puo' essere utile negli esempi pratici
    // derivata da map e join
    def flatMap[A,B](ma:M[A])(f:(A)=>M[B]) = flatten(map(ma)(f))
  }
\end{lstlisting}

Per l'esempio considererò una coppia di monadi composta con il costruttore $swap$,
vedi sezione \ref{swap_construction}

\begin{lstlisting}[style=myScalastyle, caption=Swap Constructor]
  trait SComposable[M[_],N[_]]{
    def M:Monad[M]
    def N:Monad[N]
    def swap[A](nm:N[M[A]]):M[N[A]]
  }
\end{lstlisting}

Definiamo qualche istanza di $Monad$

\begin{lstlisting}[style=myScalastyle, caption=Type class instances]
implicit val OptionMonad = new Monad[Option]{
  def map[A, B](ma: Option[A])(f: (A) => B) = ma.map(f)
  def unit[A](a: A) = Option(a)
  def flatten[A](mm: Option[Option[A]]) = mm.flatMap(identity)
}
implicit val ListMonad = new Monad[List]{
  def map[A, B](ma: List[A])(f: (A) => B) = ma.map(f)
  def unit[A](a: A) = List(a)
  def flatten[A](mm: List[List[A]]) = mm.flatten
}
\end{lstlisting}
\pagebreak
Definiamo anche la sintassi per $map$ e $flatmap$ per le monadi scala, essenzialmente
serve per poterle utilizzare dentro al ciclo $for$.
\begin{lstlisting}[style=myScalastyle, caption=Composer]
case class Mval[M[_]:Monad,A](v:M[A]){
  def map[B](f:(A)=>B) = implicitly[Monad[M]].map(v)(f)
  def flatMap[B](f:(A)=>M[B])
    = implicitly[Monad[M]].flatMap(v)(f)
}
implicit def v2M[M[_]:Monad, A](v:M[A]) = Mval(v)
def compose[M[_],N[_],A](mna:M[N[A]])
  (implicit mn:Monad[({type MN[x] = M[N[x]]})#MN])
  :Mval[({type MN[x] = M[N[x]]})#MN,A]
  = Mval[({type MN[x] = M[N[x]]})#MN,A](mna)
\end{lstlisting}
Quindi passiamo a definire istanze di $SComposable$ come $Monad$

\begin{lstlisting}[style=myScalastyle, caption=SComposable is Monad]
implicit def monadSCompose[M[_],N[_]]
  (implicit m:Monad[M], n:Monad[N], s:SComposable[M,N])
  = new Monad[({type MN[x] = M[N[x]]})#MN]{
    def map[A,B](ma:M[N[A]])(f: (A) => B)
      = m.map(ma)(n.map(_)(f))
    def unit[A](a: A) =  m.unit(n.unit(a))
    def flatten[A](mnmn:M[N[M[N[A]]]])
      = m.map(m.flatten(m.map(mnmn)(s.swap(_))))(n.flatten(_))
}
\end{lstlisting}

Ora definiamo la composizione generalizzata mediante $swap$ per le due istanze
definite
\begin{lstlisting}[style=myScalastyle, caption=Composing List]
  implicit def MonadListSComposable[M[_]]
    (implicit monadM:Monad[M])
    = new SComposable[M,List] {
    def M = monadM
    def N = implicitly[Monad[List]]
    def swap[A](nm: List[M[A]]):M[List[A]] = nm match {
      case List() => monadM.unit(List())
      case x::xs => for {
          y <- x;
          ys <- swap(xs)
        } yield y::ys
    }
  }
\end{lstlisting}
\pagebreak
\begin{lstlisting}[style=myScalastyle, caption=Composing Optional]
  implicit def MonadOptionSComposable[M[_]]
    (implicit monadM: Monad[M])
    = new SComposable[M,Option]{
      def M = monadM
      def N = implicitly[Monad[Option]]
      def swap[A](nm: Option[M[A]]) = nm match{
        case Some(ma) => monadM.map(ma)(Option(_))
        case None => monadM.unit(None)
      }
  }
\end{lstlisting}

Per finire concludo con la definizione di un semplice main per testare il codice
appena scritto
\begin{lstlisting}[style=myScalastyle, caption=Main]
  // outputs: List(Some(4), Some(5), Some(8), Some(9))
  println(for {
      i<- compose(List(Option(1), None, Option(5)))
      j<- compose(List(Option(3), Option(4)))
    } yield i+j)

  // outputs: Some(List(4, 5, 8, 9))
  println(for {
      i<- compose(Option(List(1, 5)))
      j<- compose(Option(List(3, 4)))
    } yield i+j)
\end{lstlisting}
