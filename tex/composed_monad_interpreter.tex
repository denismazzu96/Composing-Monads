\section{Composed Monad Interpreter}
\label{composed_monad_interpreter}

Ora riprendiamo il semplice interprete definito nella sezione
\ref{monadic_evaluator}, definiamolo però in termini di monade combinato.
Ricordando che la funzione di valutazione dovrà gestire l'accesso a variabili
dentro ad un $enviroments$, potrebbe produrre dell'output e richiede la gestione
degli errori.
Tutto questo è coperto dalla seguente definizione di dato astratto:
\begin{align*}
  \bm{type}\ M\ a\ =\ Env\ \to Writer\ [String]\ (Error\ a)
\end{align*}
Questa definizione può essere espressa nel seguente modo come composizione tra
monadi
\begin{align*}
  \bm{type}\ M\ a\ =\ DComp\ (Env \to)\ (SComp\ (Writer\ [String])\ Error) a
\end{align*}
Inoltre notiamo come, utilizzando $Gofer$, questa definizione risulti "brutta"
da vedere.
Potremmo ragionevolmente aspettarci di poter definire $M$ il una maniera migliore.
In un contesto adatto potremmo definire $M$ in maniera del tutto equivalente nel
seguente modo
\begin{align*}
  M\ a\ =\ (Env\ \to)\ .\ Writer\ [String]\ .\ Error\ a
\end{align*}
Utilizzando le funzioni \textit{general purpose} $left$ e $right$, vedi sezione
\ref{left_and_right}, possiamo includere la computazione di ogni componente dentro
all'intero $Monad M$ composto.
\begin{align*}
  inError &:: Error\ a \to M\ a\\
  inError &= right\ .\ right\\\\
  inReader &:: (Env \to a) \to M\ a\\
  inReader &= left\\\\
  inWriter &:: Writer\ [String]\ a \to M\ a\\
  inWriter &= right\ .\ left\\\\
\end{align*}
Possiamo anche notare da queste definizioni che si espone un concetto di associatività
all'interno di $M$, ossia l'operatore di composizione associa a destra.
Quindi le funzioni per accedere alla posizione del proprio monade all'interno di
$M$ devono percorre l'albero generato dalla composizione fino al proprio nodo.\\

La definizione di $eval$ è definita da
\begin{align*}
  eval &:: Expr \to M\ Value\\
  eval\ (Const\ v) &= [unit\ v]\ (\text{in maniera equivalente}\ [[v]])\\
  eval\ (Var\ n) &= [x\ |\ r \leftarrow inReader\ (lookup\ n),\ x \leftarrow inError\ r]\\
  eval\ (e_0 + e_1) &= [x+y\ |\ x \leftarrow eval\ e_0,\ y \leftarrow eval\ e_1]\\
  eval\ (Trace\ m\ e) &= [x\ |\ x \leftarrow eval\ e,\\
                      &\qquad\quad () \leftarrow inWriter\ (Result\ [m \doubleplus "=" \doubleplus x] ())]
\end{align*}

Osserviamo come questa versione sia finalmente la soluzione preferibile.
Propone un codice elegante e difficilmente si incorre il rischio di commettere
errori.