\section{Summary}
\label{summary}

Ora invece cerchiamo di capire dato un $ADT$
\footnote{Abstract Data Type, tipo di dato astratto}, come potremmo comporlo con
altri monadi.
Definiamo quindi una struttura generale per gli $ADT$, notiamo prima qualche
osservazione per restringere la costruzione di questi tipi di dato astratti
\begin{itemize}
  \item Non faremo distinzioni tra coppie e sequenze in quanto un $ADT$ definito
    mediante coppia e uno mediante sequenza sono equivalenti
    \begin{align*}
      \bm{data}\ A\ a = A\ (a,\ b_1,\ \dots\ ,\ b_n) \equiv \bm{data}\ B\ a = B\ a\ b_1\ \dots\ b_n
    \end{align*}
    Con i parametri $b_1,\ \dots\ ,b_n$ fissati, è facile vedere come in ogni situazione
    in cui necessito di $A$ possa utilizzare $B$ data la funzione biettiva di conversione
    \begin{align*}
      \gamma &:: A \to B\\
      \gamma\ (A\ (x,\ y_1,\ \dots\ ,\ y_n)) &= B\ x\ y_1\ \dots\ \ y_n\\
      \gamma^{-1}\ (B\ x\ y_1\ \dots\ \ y_n) &= A\ (x,\ y_1,\ \dots\ ,\ y_n)
    \end{align*}
  \item Ogni $ADT$ sarà unario per semplicità. Nota come questa non sia una restrizione
    ma solamente una semplificazione, in quanto, ogni construttore non unario
    può essere $curryfied$ e diventarlo.
    Considerando semplicemente un parametro alla volta.
\end{itemize}
La struttura che ho trovato più generale possibile per caratterizzare tutti i tipi
di $ADT$ è la seguente
\begin{align*}
  \bm{data}\ A\ a = A_0\ a{_0}{^*}\ |\ \dots\ |\ A_n\ a{_n}{^*}
\end{align*}
Dove:
\begin{itemize}
  \item $A_i$ descrivono i vari costruttore utilizzati, inoltre $|\ A_i\ \forall i\ | >0$
    non può esistere il caso in cui un tipo di dato astratto non abbia nessun costruttore
  \item $a{_i}{^*}$ è una lista di parametri (anche vuota), $\forall i,\ k.\ a{_i}{^k}$ è $a$ oppure un tipo fissato
\end{itemize}
Notiamo come i principali $ADT$ incontrati fino ad ora rispettino questa caratterizzazione:
\begin{alignat*}{2}
  Maybe\ a &= Just\ a\ &&|\ Nothing\\
  A\ a &= A_0\ a\ &&|\ A_1 \\\\
  Error\ s\ a &= Raise\ s\ &&|\ Return\ a\\
  B\ a &= A_0\ s\ &&|\ A_1\ a\\\\
  (s \to)\ a &= Reader\ (s \to a)\\
  A\ a &= A_0\ C\ a\\\\
  ZipList\ a &= Z\ [\ a\ ]\\
  A\ a &= A_0\ D\ a
\end{alignat*}
Con $B$, $C$ e $D$ definiti come:
\begin{itemize}
  \item $B \equiv (A\ a_0)$ con $a_0$ che diventa parametro fissato
  \item $C \equiv (s \to)$ per descrivere le realzioni
  \item $D \equiv \lambda x \to [\ x\ ]$ per rappresentare l'operatore di $binding$ per liste
\end{itemize}

Definiamo inoltre il \textbf{tipo di dato astratto base}, ovvero un $ADT$ che non può essere
scomposto in altri $ADT$ di inferiore struttura.
In altri termini, un $ADT$ base non è rappresentabile mediante la composizione di
più $ADT$ in ogni sua parte.\newline

\setlength{\fboxsep}{1em}
\fbox{\parbox{.9\linewidth}{Dato quindi un $ADT$ \textbf{base} possiamo
  \begin{itemize}
    \item pre-comporlo \footnote{$A$ pre-composto a $m$ se intendo la composizione di $m\ .\ A$}
     ad $m$, utilizzando il costruttore $prod$, se $m$ dispone di una funzione di aggregazione
     oppure se in $A$ non ci sono multiple apparizioni di $a$ parametro
    \item post-comporlo \footnote{$A$ post-composto a $m$ se intendo la composizione di $A\ .\ m$}
     ad $m$, utilizzando il costruttore $dorp$, se $m$ dispone di una funzione di aggregazione
     oppure se in $A$ non ci sono multiple apparizioni di $a$ parametro
    \item pre-comporlo utilizzando il costruttore $swap$, richiedendo all'arbitrario
      monade di possedere la proprietà commutativa
  \end{itemize}
}}\\\\
Inoltre possiamo osservare che anche le Monadi che per ogni costruttore utilizzano
al massimo una volta il parametro $a$ possono essere sempre composte senza limitazioni,
vedi $Maybe$ o $Error$.

Dare una dimostrazione formale di queste proprietà richiede l'esistenza di funzioni
ausiliarie di commutazione che porterebbero alla costruzione di $prod$ e $dorp$, la cui
esistenza porterebbe alla definizione di $join$.
In particolare richiediamo la proprietà commutativa $(1)$ e quella "aggregativa" $(2)$
\begin{align}
  [\ exp\ |\ x \leftarrow v,\ y \leftarrow u\ ] &= [\ exp\ |\ y \leftarrow u,\ x \leftarrow v\ ]\\
  \exists f.\ f &:: a_0 \to \dots \to a_n \to b
\end{align}
Non stiamo quindi concludendo che due monadi qualsiasi sono componibili tra di loro
dato che in questa lista riassuntiva richiediamo sempre di essere a conoscenza della
struttura di almeno uno dei due monadi.
Difatti stiamo solo ristrutturando in maniera equivalente le conclusioni del capitolo
\ref{general_framework_for_composition}.\\
Inoltre non sempre la composizione di due monadi è la modalità preferibile di composizione,
nella sezione \ref{state_transitions} vediamo un caso in cui la composizione
è semanticamente diversa al monade che stiamo cercando.

\vfill