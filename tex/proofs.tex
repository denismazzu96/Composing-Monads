\label{proofs}

\subsection{Exception is a Monad}
\label{exception_is_a_monad}
Dimostriamo che il tipo astratto $Exception$ \ref{exceptions} è un
\textit{Functor}, \textit{Premonad} e \textit{Monad}\newline

\textbf{Dim:}\newline

Dimostriamo che il tipo astratto con le operazioni di \textit{map},
\textit{unit} e \textit{join} rispetta le leggi definite per \textit{Functor},
\textit{Premonad} e \textit{Monad}, $M1\dots7$

\begin{align*}
  \textbf{data}\ Exception\ a &= Raise\ String\ |\ Return\ a \\\\
  map &:: (a \to b) \to Exception\ a \to Exception\ b\\
  map\ f\ (Raise\ s) &= Raise\ s\\
  map\ f\ (Return\ x) &= Return\ (f\ x)\\\\
  unit &:: a \to Exception\ a\\
  unit &= Return\\\\
  join &:: Exception\ (Exception\ a) \to Exception\ a\\
  join\ (Raise\ s) &= Raise\ s\\
  join\ (Return\ x) &= x
\end{align*}

Per la dimostrazione procedo per casi sui costrutti $Raise$ e $Return$

\begin{framed}
  \begin{align*}
      (Raise)\qquad
      &\quad map\ id\ (Raise\  s) && \{map\}\\
      =&\quad (Raise\ s) && \{id^{-1}\}\\
      =&\quad id\ (Raise\ s)\\\\
      (Return)\qquad
      &\quad map\ id\ (Return\ x) && \{map\}\\
      =&\quad Return\ (id\ x) && \{id\}\\
      =&\quad Return\ x
    %\caption*{$(M1)$}
  \end{align*}

  \begin{align*}
      (Raise)\qquad
      &\quad (map\ f\ .\ map\ g)\ (Raise\ s) && \{composition\}\\
      =&\quad map\ f\ (map\ g\ (Raise\ s)) && \{map\}\\
      =&\quad map\ f\ (Raise\ s) && \{map\}\\
      =&\quad Raise\ s && \{map^{-1}\}\\
      =&\quad map\ (f\ .\  g)\ (Raise\ s)\\\\
      (Return)\qquad
      &\quad (map\ f\ .\ map\ g)\ (Return\ s) && \{composition\}\\
      =&\quad map\ f\ (map\ g\ (Return\ s)) && \{map\}\\
      =&\quad map\ f\ (Return\ (g\ s)) && \{map\}\\
      =&\quad Return\ (f\ (g\ s)) && \{composition^{-1}\}\\
      =&\quad Return\ ((f\ .\ g)\ s) && \{map^{-1}\}\\
      =&\quad map\ (f\ .\  g)\ (Return\ s)
    %\caption*{$(M2)$}
  \end{align*}

  \begin{align*}
      (Raise)\qquad
      &\quad (unit\ .\ f)\ (Raise\ s) && \{composition\}\\
      =&\quad unit\ (f\ (Raise\ s)) &&\{unit\}\\
      =&\quad Return\ (f\ (Raise\ s)) && \{map^{-1}\} \\
      =&\quad map\ f\ (Return\ (Raise\ s)) && \{unit^{-1}\} \\
      =&\quad map\ f\ (unit\ (Raise\ s)) && \{composition^{-1}\} \\
      =&\quad (map\ f\ .\ unit)\ (Raise\ s)\ \\\\
      (Return)\qquad
      &\quad (unit\ .\ f)\ (Return\ s) && \{composition\}\\
      =&\quad unit\ (f\ (Return\ s)) &&\{unit\}\\
      =&\quad Return\ (f\ (Return\ s)) && \{unit^{-1}\} \\
      =&\quad Return\ (f\ (unit\ s)) && \{composition^{-1}\} \\
      =&\quad Return\ ((f\ .\ unit)\ s) && \{map^{-1}\} \\
      =&\quad (map\ f\ .\ unit)\ (Return\ s)\
    %\caption*{$(M3)$}
  \end{align*}

  \begin{align*}
      (Raise)\qquad
      &\quad (join\ .\ map\ (map\ f))\ (Raise\ s) && \{composition\}\\
      =&\quad join\ (map\ (map\ f)\ (Raise\ s)) && \{map\}\\
      =&\quad join\ (Raise\ s)  && \{join\}\\
      =&\quad Raise\ s\ && \{map^{-1}\}\\
      =&\quad map\ f\ (Raise\ s) && \{join^{-1}\}\\
      =&\quad map\ f\ (join\ (Raise\ s)) && \{composition^{-1}\}\\
      =&\quad map\ f\ .\ join\ (Raise\ s)\\\\
      (Return)\qquad
      &\quad (join\ .\ map\ (map\ f))\ (Return\ s) && \{composition\}\\
      =&\quad join\ (map\ (map\ f)\ (Return\ s)) && \{map\}\\
      =&\quad join\ (Return\ (map\ f\ s))  && \{join\}\\
      =&\quad map\ f\ s && \{join^{-1}\} \quad(*)\\
      =&\quad map\ f\ (join\ (Return\ s)) && \{composition^{-1}\}\\
      =&\quad map\ f\ .\ join\ (Return\ s)\
    %\caption*{$(M4)$}
  \end{align*}
  \begin{align*}
      (Raise)\qquad
      &\quad (join\ .\ unit)\ (Raise\ s) && \{composition\}\\
      =&\quad join\ (unit\ (Raise\ s)) && \{unit\}\\
      =&\quad join\ (Return\ (Raise\ s)) && \{join\}\\
      =&\quad Raise\ s && \{id^{-1}\}\\
      =&\quad id\ (Raise\ s) \\\\
      (Return)\qquad
      &\quad (join\ .\ unit)\ (Return\ s) && \{composition\}\\
      =&\quad join\ (unit\ (Return\ s) && \{unit\}\\
      =&\quad join\ (Return\ (Return\ s) && \{join\}\\
      =&\quad Return\ s && \{id^{-1}\}\\
      =&\quad id\ (Return\ s)
    %\caption*{$(M5)$}
  \end{align*}

  \begin{align*}
      (Raise)\qquad
      &\quad (join\ .\ map\ unit)\ (Raise\ s)) && \{composition\}\\
      =&\quad join\ (map\ unit\ (Raise\ s)) && \{map\}\\
      =&\quad join\ (Raise\ s) && \{join\}\\
      =&\quad Raise\ s\ && \{id^{-1}\}\\
      =&\quad id\ (Raise\ s) \\\\
      (Return)\qquad
      &\quad (join\ .\ map\ unit)\ (Return\ s) && \{composition\}\\
      =&\quad join\ (map\ unit\ (Return\ s)) && \{map\}\\
      =&\quad join\ (Return\ (unit\ s)) && \{unit^{-1}\}\\
      =&\quad join\ (unit\ (unit\ s)) && \{composition^{-1}\}\\
      =&\quad join\ .\ unit\ (unit\ s) && \{M5\}\\
      =&\quad id\ (unit\ s) && \{unit\}\\
      =&\quad id\ (Return\ s)
    %\caption*{$(M6)$}
  \end{align*}

  \begin{align*}
      (Raise)\qquad
      &\quad (join\ .\ map\ join)\ (Raise\ s)) && \{composition\}\\
      =&\quad join\ (map\ join\ (Raise\ s)) && \{map\}\\
      =&\quad join\ (Raise\ s) && \{join^{-1}\}\\
      =&\quad join\ (join\ (Raise\ s)) && \{composition^{-1}\}\\
      =&\quad join\ .\ join\ (Raise\ s) \\\\
      (Return)\qquad
      &\quad (join\ .\ map\ join)\ (Return\ s) && \{composition\}\\
      =&\quad join\ (map\ join\ (Return\ s)) && \{map\}\\
      =&\quad join\ (Return\ (join\ s)) && \{join\}\\
      =&\quad join\ s && \{join^{-1}\}\quad(*)\\
      =&\quad join\ (join\ (Return\ s)) && \{composition^{-1}\}\\
      =&\quad join\ .\ join\ (Return\ s)
    %\caption*{$(M7)$}
  \end{align*}
\end{framed}

Osservazione $(*)$, utilizzando la definizione inversa di $join$ richiedo che il suo
parametro $s$ sia di tipo $s::Exception$

% ------------------------------------------------------------------------------

\pagebreak
\subsection{Natural Join doesn't exist}
\label{natural_join_doesn_t_exist}
Dimostriamo che non esiste il $join$ naturale tra \textit{Monads} generici
\ref{conditions_for_composition}.
In parole, è impossibile costruire la funzione $join$ per la composizione di due
$Monads$ utilizzando esclusivamente gli operatori dei monadi.\\
Supponiamo di lavorare con due $Monads$ dati $M$ e $N$.
Nel caso più generale un termine ben tipato può essere costruito utilizzando
solo gli operatori $map$, $unit$ e $join$.
Precisamente possiamo costruire solo termini generati dalla seguenti regole
\begin{align*}
  \begin{split}
    a \overset{id}{\longrightarrow} a
  \end{split}
    \begin{split}
      \frac
        {b \overset{f}{\longrightarrow} c\quad a \overset{g}{\longrightarrow} b}
        {a \overset{f\ .\ g}{\longrightarrow} c}
    \end{split}
  % \end{split}
\end{align*}
\begin{align*}
  \begin{split}
    \frac
      {a \overset{f}{\longrightarrow} b}
      {M\ a \overset{map_M\ f}{\longrightarrow} M\ b}
  \end{split}
  \begin{split}
    \frac
      {a \overset{f}{\longrightarrow} b}
      {N\ a \overset{map_N\ f}{\longrightarrow} N\ b}
  \end{split}
\end{align*}
\begin{align*}
  \begin{split}
    a \overset{unit_M}{\longrightarrow} M\ a
  \end{split}
  \begin{split}
    a \overset{unit_N}{\longrightarrow} N\ a
  \end{split}
\end{align*}
\begin{align*}
  \begin{split}
    M\ (M\ a) \overset{join_M}{\longrightarrow} M\ a
  \end{split}
  \begin{split}
    N\ (N\ a) \overset{join_N}{\longrightarrow} N\ a
  \end{split}
\end{align*}
Queste regole comprendono il più piccolo set per poter modellare adeguatamente
le operazioni in maniera naturale.
Utilizzando questa caratterizzazione mostrerò che non c'è modo di costruire un
termine del tipo $M\ (N\ (M\ (N\ a))) \to M\ (N\ a)$ e quindi che non esiste il
$join$ naturale per la composizione dei \textit{Monads}.\newline

Per convenienza utilizzerò le seguenti convenzioni:
\begin{itemize}
  \item scriverò la stringa $MNMNX$ al posto dell'espressione $M\ (N\ (M\ (N\ X)))$
        ,inoltre per l'i-esimo elemento di una qualsiasi stringa $X$ utilizzerò
        la notazione $(X)_i$, ad esempio, $(MNMMX)_1 = M$, $(MNMMX)_5 = X$ e $(MNMMX)_6 = \emptyset$
  \item userò la notazione $rd\ X$ per la stringa ottenuta rimuovendo tutti i
        duplicati adiacenti da $X$, per esempio, $rd\ MMNMNNX\ =\ MNMNX$
  \item $\req{X}$ è l'operatore che ritorna $|rd\ X|$, ossia la cardinalità di $rd\ X$

\end{itemize}
Notiamo quindi che se le regole base definissero solo funzioni del tipo
\begin{align*}
  X \to Y \implies
  \begin{cases}
   \req{X}\ <\ \req{Y} &if\ (X)_1 \neq (Y)_1\\
   \req{X}\ \leq\ \req{Y} &if\ (X)_1 = (Y)_1
  \end{cases}
  && (*)
\end{align*}
allora potremmo conseguire che la funzione $join$ non è ottenibile essendo che
la funzione del tipo $MNMNX \to MNX$ ha come proprietà
\begin{align*}
  (MNMNX)_1 = M = (MNX)_1 \land
  \req{MNMNX}\ >\ \req{MNX}
\end{align*}
E quindi non rispetta $(*)$.\newline

\textbf{Dim:}\newline

Procedo per induzione sull'altezza $h$ dell'albero di derivazione generato dalle
regole definite.\newline

\textbf{Caso base:} $(h = 0)$ nessuna regola mi permette di procedere
senza eseguire almeno un passo quindi il predicato $X \to_0 Y$ è falso.
Implica che la proprietà $(*)$ è vacuamente vera.\newline

\textbf{Caso induttivo:} $(h \to h+1)$\\
La proprietà $(*)$ da dimostrare diventa quindi
\begin{align*}
  X \to_{h+1} Y \implies
  \begin{cases}
   \req{X}\ <\ \req{Y} &if\ (X)_1 \neq (Y)_1\\
   \req{X}\ \leq\ \req{Y} &if\ (X)_1 = (Y)_1
  \end{cases}
  && (**)
\end{align*}
L'ipotesi induttiva invece è così definita
\begin{align*}
  X \to_{\leq h} Y \land (X)_1 \neq (Y)_1 \implies \req{X}\ <\ \req{Y} &&(IP1)\\
  X \to_{\leq h} Y \land (X)_1 = (Y)_1 \implies \req{X}\ \leq\ \req{Y} &&(IP2)
\end{align*}
Procedo per casi per ogni regola definita
\begin{itemize}
  \item \textbf{[id]}
    \begin{align*}
        \frac{\req{X}\ \leq\  \req{X}}{X \overset{id}{\longrightarrow}_{h+1} X}
    \end{align*}
      In questo caso è ovvio che $(X)_1 = (X)_1$\\
      E dato che l'operatore $\leq$ gode della proprietà riflessiva quindi ho
      dimostrato $(**)$ per l'identità
  \item \textbf{[composition]}
    \begin{align*}
        \frac{
        \exists Z.
        \quad \overset{(1)}{Z \overset{f}{\longrightarrow}_{\leq h} Y}
        \quad \overset{(2)}{X \overset{g}{\longrightarrow}_{\leq h} Z}
        }{X \overset{f\ .\ g}{\longrightarrow}_{h+1} Y}
    \end{align*}
    Procediamo per casi su $(1)$ e $(2)$:
    \begin{itemize}
      \item $(Z)_1 = (Y)_1 \land (X)_1 = (Z)_1$, quindi applico l'ipotesi induttiva
        $(IP2)$ a $(1)$ e $(2)$ e ottengo che $\req{Z}\ \leq\ \req{Y} \land
        \req{X}\ \leq\ \req{Z}$.\\
        Dato che l'operatore $\leq$ gode della proprietà
        riflessiva ho che vale quindi $\req{X}\ \leq\ \req{Y}$.\\
        Per concludere basta osservare che $(Z)_1 = (Y)_1 \land (X)_1 = (Z)_1
        \implies (X)_1 = (Y)_1$ e quindi devo provare la proprietà $\req{X}\ \leq\ \req{Y}$
        in $(**)$.\\
      \item $(Z)_1 = (Y)_1 \land (X)_1 \neq (Z)_1$, quindi applico l'ipotesi induttiva
        $(IP2)$ a $(1)$ e $(IP1)$ a $(2)$ e ottengo che $\req{Z}\ \leq\ \req{Y} \land
        \req{X}\ <\ \req{Z}$.\\
        Concatenando $(1)$ e $(2)$ ottengo $\req{X}\ <\ \req{Z}\ \leq\ \req{Y}$
        che implica $\req{X}\ <\ \req{Y}$.\\
        Per concludere basta osservare che $(Z)_1 = (Y)_1 \land (X)_1 \neq (Z)_1
        \implies (X)_1 \neq (Y)_1$ e quindi devo provare la proprietà $\req{X}\ <\ \req{Y}$
        in $(**)$.\\
      \item $(Z)_1 \neq (Y)_1 \land (X)_1 = (Z)_1$, duale al caso precedente.\\
      \item $(Z)_1 \neq (Y)_1 \land (X)_1 \neq (Z)_1$, quindi applico l'ipotesi induttiva
        $(IP1)$ a $(1)$ e $(2)$ e ottengo che $\req{Z}\ <\ \req{Y} \land
        \req{X}\ <\ \req{Z}$.\\
        Dato che l'operatore $<$ gode della proprietà
        riflessiva ho che vale quindi $\req{X}\ <\ \req{Y}$.\\
        Ho concluso dimostrando $(**)$ dato che senza interessarci a quanto vale
        l'uguaglianza tra $(X)_1$ e $(Y)_1$ ho comunque già valida l'ipotesi più
        forte, ovvero $\req{X}\ <\ \req{Y}$.\\
    \end{itemize}
    Per tutti i quattro casi vale $(**)$ quindi ho dimostrato $(**)$ per la composizione.
    \item \textbf{[map]} dimostro per $map_M$ e ottengo anche il duale $map_N$
      \begin{align*}
          \frac{
            \overset{(1)}{X \overset{f}{\longrightarrow}_{\leq h} Y}
          }{MX \overset{map_M\ f}{\longrightarrow}_{h+1} MY}
      \end{align*}
      Ora notiamo che vale $(MX)_1 = M = (MY)_1$ quindi la proprietà da dimostrare
      in tutti i casi è la meno restrittiva, ossia $\req{MX} \leq\ \req{MY}$.\\
      Procedo per casi su $(1)$:
      \begin{itemize}
        \item $(X)_1 = (Y)_1$, quindi applico l'ipotesi induttiva $(IP2)$ e
          ottengo $\req{X}\ \leq\ \req{Y}$.\\
          Ora mi trovo davanti a due sottocasi:
          \begin{itemize}
            \item $(X)_1 = M (\implies (Y)_1 = M)$, allora ho che
              \begin{align*}
              \req{MX} = \req{MMX'} = \req{MX'} = \req{X}\\
              \req{MY} = \req{MMY'} = \req{MY'} = \req{Y}
              \end{align*}
              quindi data l'ipotesi induttiva ottengo che
            \begin{equation*}\req{MX} = \req{X} \leq_{ip. ind}\ \req{Y} = \req{MY}\end{equation*}
            \item $(X)_1 = N (\implies (Y)_1 = N)$, allora ho che
              \begin{align*}
              \req{MX} = \req{MNX'} \overset{(a)}{=} 1 + \req{NX'} = 1 + \req{X}\\
              \req{MY} = \req{MNY'} \overset{(a)}{=} 1 + \req{NY'} = 1 + \req{Y}
              \end{align*}
              quindi data l'ipotesi induttiva ottengo che
            \begin{equation*}\req{MX} = 1 + \req{X} \leq_{ip. ind}\ 1 + \req{Y} = \req{MY}\end{equation*}
              Il passaggio $(a)$ è giustificato dal fatto che $\req{MNX'} = |rd\ MNX'| = 1 + |rd\ NX'|$
          \end{itemize}
        \item $(X)_1 \neq (Y)_1$, quindi applico l'ipotesi induttiva $(IP1)$ e
          ottengo $\req{X}\ <\ \req{Y}$.\\
          Ora mi trovo davanti a due sottocasi:
          \begin{itemize}
            \item $(X)_1 = M (\implies (Y)_1 = N)$, allora ho che
              \begin{align*}
              \req{MX} = \req{MMX'} = \req{MX'} = \req{X}\\
              \req{MY} = \req{MNY'} = 1 + \req{NY'} = 1 + \req{Y}
              \end{align*}
              quindi data l'ipotesi induttiva ottengo che
            \begin{equation*}\req{MX} = \req{X} <_{ip. ind}\ \req{Y} <\ 1 + \req{Y} = \req{MY}\end{equation*}
            \item $(X)_1 = N (\implies (Y)_1 = M)$, allora ho che
              \begin{align*}
              \req{MX} = \req{MNX'} = 1 + \req{NX'} = 1 + \req{X}\\
              \req{MY} = \req{MMY'} = \req{MY'} = \req{Y}
              \end{align*}
              quindi data l'ipotesi induttiva ottengo che
            \begin{align*}
              &\quad\req{MX} = 1 + \req{X} <_{ip. ind}\ 1 + \req{Y}\\
              &\implies 1 + \req{X} \leq\ \req{Y} = \req{MY}
            \end{align*}
          \end{itemize}
          Concludo quindi di aver dimostrato $(**)$ per l'operatore $map$.
      \end{itemize}
      \item \textbf{[unit]} dimostro per $unit_M$ e ottengo anche il duale $unit_N$
        \begin{equation*}X \overset{unit_M}{\longrightarrow_{h+1}} MX\end{equation*}
        Mi trovo davanti a due casi:
        \begin{itemize}
          \item $(X)_1 = M \implies \req{MX} = \req{MMX'} = \req{MX'} = \req{X}$\\
            Per riflessività di $\leq$ ottengo
            \begin{equation*}\req{X} = \req{MX} \leq\ \req{MX}\end{equation*}
            Inoltre essendo che $(X)_1 = M = (MX)_1$ allora ho dimostrato $(**)$
            per questo caso.
          \item  $(X)_1 = N \implies \req{MX} = \req{MNX'} = 1 + \req{NX'} = 1 + \req{X}$\\
            Quindi ottengo
            \begin{equation*}\req{X} <\ 1 + \req{X} = \req{MX}\end{equation*}
            Inoltre essendo che $(X)_1 = N \neq M = (MX)_1$ allora ho dimostrato $(**)$
            per questo caso.
        \end{itemize}
        Concludo quindi di aver dimostrato $(**)$ per l'operatore $unit$.
      \item \textbf{[join]} dimostro per $join_M$ e ottengo anche il duale $join_N$
        \begin{equation*}MMX \overset{join_M}{\longrightarrow_{h+1}} MX\end{equation*}
        Per l'operatore di $join$ concludo direttamente osservando le due proprietà
        \begin{align*}
          (MMX)_1 = M = (MX)_1\\
          \req{MMX} = \req{MX} \leq\ \req{MX}
        \end{align*}
        Quindi concludo anche per quest'ultimo caso che la proprietà $(**)$ è valida
        per l'operatore di $join$.
\end{itemize}

Questa dimostrazione prova quindi che, con la grammatica sopra definita,
la funzione $join$ non è ottenibile essendo funzioni del tipo $MNMNX \to MNX$
hanno come proprietà
\begin{align*}
  (MNMNX)_1 = M = (MNX)_1 \land
  \req{MNMNX}\ >\ \req{MNX}
\end{align*}
Non ottenibili in questo contesto dato che non soddisfano $(*)$.\newline


\textbf{Conseguenze:} come conseguenze possiamo notare che gli operatori $map$ e
$unit$ possono essere generati in quanto rispettano $(**)$
\begin{align*}
  (MNX)_1 = M = (MNX)_1\  \land\  &\req{MNX} \leq \req{MNX} &&(map_M\ .\ map_N\ f)\\
  &\req{X} < \req{MNX} &&(unit_M\ .\ unit_N)
\end{align*}
Mentre nessuno tra $prod$, $dorp$ e $swap$ può essere costruito.
% ------------------------------------------------------------------------------

\pagebreak
\subsection{Monad Comprehensions Equivalence}
\label{monad_comprehensions_equivalence}
Dimostriamo la seguente proprietà, mostrata nella sezione \ref{monad_comprehensions}
\begin{align*}
    &{[\ v_0 + v_1\ |\ v_0 \leftarrow eval\ e_0\ env,\ v_1 \leftarrow eval\ e_1\ env\ ]} = \\
    &bind\ (eval\ e_0\ env) (\lambda\ v_0 \to bind\ (eval\ e_1\ env) (\lambda\ v_1 \to unit (v_0 + v_1)))
\end{align*}

Per semplicità utilizzerò le seguenti definizioni durante la dimostrazione,
essendo equazioni non modifico la semantica della proprietà da dimostrare

\begin{align*}
    E_i = eval\  e_i\  env \qquad \forall i \in {[0,\ 1]}
\end{align*}
La proprietà da dimostrare diventa
\begin{align*}
    &{[\ v_0 + v_1\ |\ v_0 \leftarrow E_0,\ v_1 \leftarrow E_1\ ]} = \\
    &bind\ E_0\ (\lambda\ v_0 \to bind\ E_1\ (\lambda\ v_1 \to unit\ (v_0 + v_1)))
\end{align*}

\textbf{Dim:}\newline

Prima di cominciare bisogna definire una regola che segue direttamente da
un'osservazione delle regole $(M5)$ e $(M6)$
\begin{align*}
    map\ (f\ .\ unit)\ &=\ map\ f\ .\ unit \qquad&& (Sub)
  \end{align*}
Questa proprietà si prova facilmente con la seguente dimostrazione
\begin{framed}
\begin{align*}
    &\quad map\ (f\ .\ unit)\ && \{M2\}\\
    =&\quad map\ f\ .\ map\ unit &&\{(*)\}\\
    =&\quad map\ f\ .\ unit
\end{align*}

\begin{align*}
    (*)\ join\ .\ unit\ \stackrel{(M5)}{=} id \stackrel{(M6^{-1})}{=} join\ .\ map\ unit\\
    \Rightarrow unit\ =\ map\ .\ unit
\end{align*}


Ora invece possiamo procedere alla dimostrazione principale
\begin{align*}
    &\quad{[\ v_0 + v_1\ |\ v_0 \leftarrow E_0,\ v_1 \leftarrow E_1\ ]} && \{joinComp\}\\
    =&\quad join\ {[\ {[\ v_0 + v_1\ |\ v_1 \leftarrow E_1\ ]}\ |\ v_0 \leftarrow E_0 \ ]} && \{mapComp\}\\
    =&\quad join\ {[\ map\ (v_0+)\ E_1\ |\ v_0 \leftarrow E_0 \ ]} && \{mapComp\}\\
    =&\quad join\ (map\ (\lambda\ v_0 \to\\
     &\quad \qquad map\ (v_0+)\ E_1)\ E_0 ) && \{id\}\\
    =&\quad join\ (map\ (\lambda\ v_0 \to\\
     &\quad \qquad (map\ (v_0+)\ .\ id)\ E_1)\ E_0 ) && \{M5^{-1}\}\\
    =&\quad join\ (map\ (\lambda\ v_0 \to\\
     &\quad \qquad (map\ (v_0+)\ .\ join\ .\ unit)\ E_1)\ E_0 ) && \{M4^{-1}\}\\
    =&\quad join\ (map\ (\lambda\ v_0 \to\\
     &\quad \qquad (join\ .\ map\ (map\ (v_0+))\ .\ unit)\ E_1)\ E_0 ) && \{Sub\}\\
    =&\quad join\ (map\ (\lambda\ v_0 \to\\
     &\quad \qquad (join\ .\ map\ (map\ (v_0+)\ .\ unit))\ E_1)\ E_0 ) && \{M4^{-1}\}\\
    =&\quad join\ (map\ (\lambda\ v_0 \to\\
     &\quad \qquad (join\ .\ map\ (unit\ .\ (v_0+)))\ E_1)\ E_0 ) && \{composition\}\\
    =&\quad join\ (map\ (\lambda\ v_0 \to\\
     &\quad \qquad join\ (map\ (unit\ .\ (v_0+))\ E_1))\ E_0 ) && \{bind^{-1}\}\\
    =&\quad join\ (map\ (\lambda\ v_0 \to\\
     &\quad \qquad bind\ E_1\ (unit\ .\ (v_0+)))\ E_0 ) && \{bind^{-1}\}\\
    =&\quad bind\ E_0\ (\lambda\ v_0 \to bind\ E_1\ (unit\ .\ (v_0+))) && \{composition\}\\
    =&\quad bind\ E_0\ (\lambda\ v_0 \to bind\ E_1\ (\lambda\ v_1 \to unit\ (v_0+v_1))) && \{E_0\ e\ E_1\}\\
    =&\quad bind\ (eval\ e_0\ env) \lambda\ v_0 \to \\
     &\quad \qquad \qquad bind\ (eval\ e_1\ env) \lambda\ v_1 \to \\
     &\quad \qquad \qquad \qquad unit\ (v_0+v_1)
  \end{align*}
\end{framed}