\section{Introduction}

Recentemente, il concetto di Monade è diventato un importante e pratico
strumento per i linguaggi funzionali.
La ragionde di questo dipende dal fatto che i \textit{Monads} propongono un
framework uniforme per descrivere una enorme set di caratteristiche tipicamente
impoerative all'interno di un sistema puro.\\
Nella sezione \ref{a_simple_evaluator} vedremo una possibile implementazione per
una funzione di valutazione di una semplice grammatica, esplicitando tutta la
computazione necessaria.\\
L'introduzione ai \textit{Monads} è definita nel capitolo \ref{monads}, dove
definisco la struttura di un Monade e ne descrivo le proprietà.\\
Successivamente, la sezione \ref{monadic_evaluator}, descrive lo stesso
interprete in chiave monadica, esponendo i vantaggi dell'applicazione delle
\textit{Monads} su questo semplice esempio.\\
La sezione \ref{composing_monads} pone le basi e descrive la caratterizzazione
generale del concetto di \textit{Composing Monads}, questa sezione non entra nel
dettaglio dell'algebra astratta ad alto livello, ma ragione su una astrazione
di un linguaggio funzionale puro. Questa è la sezione principale di questo
report, definendo formalmente le tre classi di possibili tecniche per combinare
\textit{Monads} in maniera arbitraria.\\
Passiamo poi a definire nella sezione \ref{general_framework_for_composition}
un sistema per generalizzare la composizione dei monadi.\\
L'ultima sezione (capitolo \ref{composed_monad_interpreter}) infine implementa
i meccanismi definiti nella sezione \ref{composing_monads} nello stesso esempio
visto nei primi due capitoli.\newline

Per questo report non assumo nessun linguaggio di programmazione, gli esempi e
la sintassi utilizzata fanno riferimento a \textit{Gofer}
\footnote{\textit{Gofer}, un piccolo, sperimentale e semplice linguaggio
puramente funzionale}.\newline

In appendice \ref{proofs} definisco le dimostrazioni che ritengo più importanti
e istruttive. Alcune dimostrazioni semplici e/o limitate sono definite tra le
righe, solitamente delimitate da un semplice box.
Mentre in appendice \ref{state_transformer} mostro un esempio di come un
\textit{composed monad} non riesce ad esprimere il concetto di
\textit{State Transitions}, per questo esempio noteremo come un'implementazione
ad-hoc sia diversa dall'equivalente rappresentazione mediante composizione.


\newpage